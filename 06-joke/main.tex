\documentclass{article} %указываем класс
%\documentclass[russian,utf8]{eskdtext} %указываем класс

%подключаются необходимые пакеты
\usepackage[utf8]{inputenc} %кодировка текста
\usepackage[russian]{babel} %поддержка русского языка
\usepackage{indentfirst} %у русских принято делать 
                         %отступ у первого абзаца

%глобальная настройка командами
\title{Методы определения плотности мух} %название
\author{Д.~И.~Хл\'{о}фос} %автор
%\date{\today} %дата создания - сегодня

\begin{document} %начало тела документа
    \maketitle   %титульный лист
    \begin{abstract}
        В статье рассматриваются теория и основной практический метод оценки плотности мух в замкнутых объемах пространства.
    \end{abstract}
    \tableofcontents %оглавление,содержание
    \newpage
    
    \section{Определение плотности мух}
    Плотность мух $\rho$ определяется по формуле
    \begin{equation}\label{eq:fliesDensity}
        \rho = \frac{n}{V},
    \end{equation}
    где $n$ --- количество мух в объеме $V$.

    Следовательно, оценка плотности мух порождает две проблемы: оценку количества мух и оценку объема.
    
    \subsection{Оценка количества мух}
    Практический учёт мух (оценка $n$ в формуле \ref{eq:fliesDensity}) является сложной задачей. Наблюдатель только своим присутствием вносит существенные погрешности: мухи разлетаются. 
    
    В больших замкнутых пространствах хорошие результаты показывает мухобоечный метод. Несомненым достоинством метода является малая погрешность и возможность контрольного учёта по звуку.
    
    Существенным недостатком являются высокие требование к физической подготовке учёного.
    
    \subsection{Оценка объемов}
    Объём комнаты с мухами можно оценить по формуле 
    \[V=h\times w\times l,\]
    где $h$ --- высота, $w$ --- ширина, а $l$ --- длина комнаты с мухами.
    
    Современные строительные нормы утверждают, что относительная погрешность этого метода не так и высока.
    
    \section{Актуальные проблемы оценки плотности мух}
    Задача оценка плотности мух в произвольной области пространства до сих пор не решена.
    
    Многолетний опыт наблюдения за мухами позволяет утверждать, что в замкнутых пространствах в состоянии покоя мухи сидят на стенах и окнах, при этом в центре комнаты плотность мух равна нулю. Плотность обеспокоенных мух носит вероятносный характер.

    Также отмечается странный факт, что кластеры плотности мух образуются в местах скопления хлебных крошек и сладостей.
    
    \section{Заключение}
    Гуманизация современного общества требует использования мухобоечного метода только в лабораторных условиях.
\end{document} %конец тела документа
