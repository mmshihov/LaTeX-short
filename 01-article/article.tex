\documentclass{article} %указываем класс

%подключаем необходимые пакеты
\usepackage[utf8]{inputenc} %кодировка текста
\usepackage[russian]{babel} %поддержка русского языка
\usepackage{indentfirst} %у русских принято делать 
                         %отступ у первого абзаца

%глобальная настройка командами
\title{Структура \LaTeX\ документа} %название
\author{М.~М.~Шихов} %автор
\date{\today} %дата создания документа

\begin{document} %начало тела документа
    \maketitle %титульный лист
    \newpage
    \begin{abstract}
        Текст  аннотации.
    \end{abstract}
    
    \newpage
    \tableofcontents %оглавление,содержание
    
    \newpage

    \section{Заглавие первой секции}
    Текст первой секции, абзац первый.
    \begin{equation}
        \label{eq:pifagorv2}
        x=\sqrt{a^2+b^2}
    \end{equation}
    
    Первая ссылка (\ref{eq:pifagorv2})
    

    \section{Заглавие первой секции}
    Текст первой секции, абзац первый.
    \begin{equation}
        \label{eq:pifagor}
        x=\sqrt{a^2+b^2}
    \end{equation}
    
    Первая ссылка (\ref{eq:pifagor})
    
    \section{Заглавие первой секции}
    Текст первой секции, абзац первый.
    
    Текст первой секции, абзац второй.
    \subsection{Заглавие первой подсекции}
    Текст первой подсекции.
    
    Из книг по \LaTeXe\ можно рекомендовать \cite{bib:cotelnikov,bib:baldin}. Про \TeX\ от автора \cite{bib:knuth:AllAbout}.
    
    
    \begin{thebibliography}{99}
        \bibitem{bib:cotelnikov} И.Котельников. \LaTeX\ по-русски / И.Котельников, П.Чеботаев --- Новосибирск:Сибирский хронограф,2009.---496 с.
        \bibitem{bib:baldin} Е.М.Балдин. Компьютерная типография \LaTeX\ / Е.М.Балдин. --- СПб.:БХВ-Петербург,2008. --- 304 c.
        \bibitem{bib:knuth:AllAbout}Д.Э.Кнут. Все про \TeX\ /Д.Э.Кнут --- М.: Вильямс,2003. ---560
    \end{thebibliography}
\end{document} %конец документа
