\usepackage{etex}

\mode<article>{\usepackage{fullpage}}
\mode<presentation>{
    \usetheme{Madrid}
    \useoutertheme{shadow}
} 

\usepackage[russian]{babel}
\usepackage[utf8]{inputenc}
\usepackage{graphicx}
\usepackage[all]{xy}
\entrymodifiers={++[o][F-]}

\title[Графы в beamer]{Графы в beamer}
\date{\today}
\author[М.~М.~Шихов]{Михаил Шихов \\ \texttt{\underline{kafevm@mail.ru}}}


\begin{document}


\mode<article>{\maketitle\tableofcontents}

\frame<presentation>{\titlepage}
\begin{frame}<presentation>[allowframebreaks]
    \frametitle{Содержание}
    \tableofcontents
\end{frame}


\section{Основные определения}

\subsection{Графы}

%uncover работает для вершин, а only работает для стрелок!!!
\begin{frame}
    \frametitle{Виды графов}
    \begin{columns}
        \column{.3\textwidth}
            \begin{block}{Неориентированный}
                \[
                    {\xymatrix{
                        a \ar@{-}[r] \ar@{-}[d]
                            &c \ar@{-}[d] \ar@{-}@(u,r)[]
                                \\
                        b \ar@{-}[ur] \ar@{-}[r]
                            &d
                    }}
                \]
                $[c,d]$ --- \alert{ребро}
            \end{block}
        
        \column{.3\textwidth}
            \begin{block}{Ориентированный}
                \[
                    {\xymatrix{
                        a \ar@{->}[r]
                            &c \ar@{->}@(u,r)[]
                                \\
                        b  \ar@{->}[u] \ar@{->}[ur] \ar@{->}[r]
                            &d \ar@{->}[u]
                    }}
                \]
                $(d,c)$ --- \alert{дуга}
            \end{block}
        \column{.3\textwidth}
            \begin{block}{Мультиграф}
            
                \[
                    {\xymatrix{
                        a \ar@{-}[r] \ar@{-}@/^/[d]
                            &c \ar@{-}[d] \ar@{-}@(u,r)[]
                                \\
                        b \ar@{-}[ur] \ar@{-}@/^/[r] \ar@{-}@/^/[u]
                            &d \ar@{-}@/^/[l]
                    }}
                \]
                $[c,c]$ --- \alert{петля}
            \end{block}
    \end{columns}
\end{frame}


\begin{frame}
    \frametitle{Помеченные (нагруженные) графы}

    \begin{columns}
        \column{.3\textwidth}
            \begin{block}{Помеченный граф}
                \[
                    {\xymatrix{
                        \ar@{-}[r]^{a}
                            &\ar@{-}@(u,r)[]|-{b}
                                \\
                        \ar@{-}[u]^{c} \ar@{-}[ur]|-{d} \ar@{-}[r]_{e}
                            &\ar@{-}[u]_{f}
                    }}
                \]
            \end{block}
        \column{.3\textwidth}
            \begin{block}{Взвешенный орграф}
                \[
                    {\xymatrix{
                        \ar@{->}[r]^{1}
                            &\ar@{->}@(u,r)[]|-{0}
                                \\
                        \ar@{->}[u]^{2} \ar@{->}[ur]|-{3} \ar@{->}[r]_{4}
                            &\ar@{->}[u]_{5}
                    }}
                \]
            \end{block}
        \column{.3\textwidth}
            \begin{block}{Взвешенный орграф}
                \[
                    {\xymatrix{
                        a \ar@{->}[r]^{1}
                            &c \ar@{->}@(u,r)[]|-{0}
                                \\
                        b \ar@{->}[u]^{2} \ar@{->}[ur]|-{3} \ar@{->}[r]_{4}
                            &d \ar@{->}[u]_{5}
                    }}
                \]
            \end{block}
    \end{columns}
\end{frame}


\subsection{Элементы графов и их свойства}

\begin{frame}
    \frametitle{Смежность и инцидентность}

    \[
        {\xymatrix{
            \uncover<1,2,3,4,6>{1}
            \ar@{-}[r]
                &\uncover<1,6>{3}
                \ar@{-}@(u,r)[]
                    \\
            \uncover<1,2,3,4,5,6>{2}
            \only<1,5,6>{\ar@{-}[u]}\only<2,3,4>{\ar@{:}[u]} 
            \ar@{-}[ur] 
            \only<1,2,4,6>{\ar@{-}[r]}\only<3,5>{\ar@{:}[r]}
                &\uncover<1,3,5,6>{4}
                \ar@{-}[u]
        }}
    \]
    
    \begin{itemize}
        \item<2,6> Вершины $1$ и $2$ --- \alert<2>{смежны}.
        \item<3,6> Ребра $[1,2]$ и $[2,4]$ --- \alert<3>{смежны}.
        \item<4,6> Ребро $[1,2]$ \alert<4>{инцидентно} вершинам $1$ и $2$.
        \item<5,6> Вершины $2$ и $4$ \alert<5>{инцидентны} ребру $[2,4]$.
    \end{itemize}
    \uncover<6>{Смежными бывают однородные элементы, а инцидентными разнородные.}
\end{frame}


\begin{frame}
    \frametitle{Степени вершин}

    \begin{columns}
        \column{.47\textwidth}
            \begin{block}{Неориентированный граф}
                \[
                    {\xymatrix{
                        a   \only<1,2,4>{\ar@{-}[r]}\only<3>{\ar@{:}[r]}
                            \only<1,3,4>{\ar@{-}[d]}\only<2>{\ar@{:}[d]}
                            &c  \only<1,2,4>{\ar@{-}[d]}\only<3>{\ar@{:}[d]}
                                \only<1,2,4>{\ar@{-}@(u,r)[]}\only<3>{\ar@{:}@(u,r)[]}
                                \\
                        b   \only<1,4>{\ar@{-}[ur]}\only<2,3>{\ar@{:}[ur]}  
                            \only<1,3,4>{\ar@{-}[r]}\only<2>{\ar@{:}[r]}
                            &d
                    }}
                \]
                \begin{itemize}
                    \item<2,4> \alert<2>{Степень} вершины $b$=$3$;
                    \item<3,4> \alert<3>{Степень} вершины $c$=$4$.
                \end{itemize}
                \uncover<4>{Сумма степеней всех вершин графа в два раза больше количества ребер.}
            \end{block}
        
        \column{.47\textwidth}
            \begin{block}{Ориентированный граф}
                \[
                    {\xymatrix{
                        a \ar@{->}[r]
                            &c \ar@{->}@(u,r)[]
                                \\
                        b   \only<1,3,4>{\ar@{->}[u]}\only<2>{\ar@{:>}[u]}
                            \only<1,2,4>{\ar@{<-}[ur]}\only<3>{\ar@{<:}[ur]}
                            \only<1,3,4>{\ar@{->}[r]}\only<2>{\ar@{:>}[r]}
                            &d \ar@{->}[u]
                    }}
                \]
                \begin{itemize}
                    \item<2,4> Полустепень \alert<2>{исхода} $b$=$2$;
                    \item<3,4> Полустепень \alert<3>{захода} $b$=$1$.
                \end{itemize}                
                \uncover<4>{Сумма соответствующих полустепеней всех вершин графа равна количеству ребер.}
            \end{block}
    \end{columns}
\end{frame}

\begin{frame}
    \frametitle{Подграф, полный граф и дополнение графа}

    \begin{columns}
        \column{.3\textwidth}
            \begin{block}{$G=\langle V,P\rangle,n=5$}
                \[
                    {\xymatrix{
                        *{} 
                            & \ar@{-}[dr] \ar@{-}[dl] 
                                &*{}
                                    \\
                        {} 
                            &*{}
                                & \ar@{-}[d] 
                                    \\
                        \ar@{-}[u] 
                            &*{}
                                & \ar@{-}[ll]
                    }}
                \]
            \end{block}
        
        \column{.3\textwidth}
            \begin{block}{Полный граф $n=5$}
                \[
                    {\xymatrix{
                        *{} 
                            & \ar@{-}[dr] \ar@{-}[ddr] \ar@{-}[ddl] \ar@{-}[dl] 
                                &*{}
                                    \\
                        {} 
                            &*{}
                                & \ar@{-}[d] \ar@{-}[dll] \ar@{-}[ll] 
                                    \\
                        \ar@{-}[u] 
                            &*{}
                                & \ar@{-}[ll] \ar@{-}[ull] 
                    }}
                \]
            \end{block}
        \column{.3\textwidth}
            \begin{block}{$\overline{G}, n=5$}
                \[
                    {\xymatrix{
                        *{} 
                            & \ar@{-}[ddr] \ar@{-}[ddl] 
                                &*{}
                                    \\
                        {} 
                            &*{}
                                & \ar@{-}[dll] \ar@{-}[ll] 
                                    \\
                        {} 
                            &*{}
                                & \ar@{-}[ull] 
                    }}
                \]
            \end{block}
    \end{columns}
    
    Количество ребер полного графа 
    \[
        m=\frac{n(n-1)}{2}.
    \]
\end{frame}


\subsection{Маршруты в графах}

\begin{frame}
    \frametitle{Маршрут $(v_1,v_{n+1})$}
    \framesubtitle{$v_1,p_1,v_2,p_2,\ldots,p_{n},v_{n+1}$}

    \begin{columns}
        \column{.4\textwidth}
            \begin{block}{$G=\langle V,P\rangle,n=5$}
                \[
                    {\xymatrix{
                        1 \ar@{-}[dd]_{a} \ar@{-}[rr]^{d} \ar@{-}[dr]_{b}
                            &*{}
                                &4 \ar@{-}[dd]^{h}
                                    \\
                        *{} 
                            &3 \ar@{-}[ur]_{f} \ar@{-}[dr]^{g}
                                &*{}
                                    \\
                        2  \ar@{-}[ur]^{c} \ar@{-}[rr]_{e}
                            &*{}
                                &5
                    }}
                \]
            \end{block}
        
        \column{.5\textwidth}
            \begin{block}{Маршрут $(1,5)$ \\ $1,b,3,c,2,a,1,b,3,g,5$}
                \[
                    {\xymatrix{
                        1 \ar@{:}[dd]_{a} \ar@{-}[rr]^{d} \ar@{:}[dr]_{b}
                            &*{}
                                &4 \ar@{-}[dd]^{h}
                                    \\
                        *{} 
                            &3 \ar@{-}[ur]_{f} \ar@{:}[dr]^{g}
                                &*{}
                                    \\
                        2  \ar@{:}[ur]^{c} \ar@{-}[rr]_{e}
                            &*{}
                                &5
                    }}
                \]
            \end{block}
    \end{columns}
\end{frame}

\begin{frame}
    \frametitle{Цепь и путь}

    \begin{columns}
        \column{.45\textwidth}
            \begin{block}{Цепь $(1,2)$\\$1,b,3,g,5,h,4,f,3,c,2$}
                \[
                    {\xymatrix{
                        1 \ar@{-}[dd]_{a} \ar@{-}[rr]^{d} \ar@{:}[dr]_{b}
                            &*{}
                                &4 \ar@{:}[dd]^{h}
                                    \\
                        *{} 
                            &3 \ar@{:}[ur]_{f} \ar@{:}[dr]^{g}
                                &*{}
                                    \\
                        2  \ar@{:}[ur]^{c} \ar@{-}[rr]_{e}
                            &*{}
                                &5
                    }}
                \]
            \end{block}
        
        \column{.45\textwidth}
            \begin{block}{Путь $(1,4)$\\$1,b,3,g,5,e,2,c,3,f,4$}
                \[
                    {\xymatrix{
                        1 \ar@{<-}[dd]_{a} \ar@{<-}[rr]^{d} \ar@{:>}[dr]_{b}
                            &*{}
                                &4 \ar@{<-}[dd]^{h}
                                    \\
                        *{} 
                            &3 \ar@{:>}[ur]_{f} \ar@{:>}[dr]^{g}
                                &*{}
                                    \\
                        2  \ar@{:>}[ur]^{c} \ar@{<:}[rr]_{e}
                            &*{}
                                &5
                    }}
                \]
            \end{block}
    \end{columns}    
\end{frame}

\begin{frame}
    \frametitle{Простая цепь и простой путь}

    \begin{columns}
        \column{.45\textwidth}
            \begin{block}{Простая цепь $(1,2)$\\$1,b,3,g,5,e,2$}
                \[
                    {\xymatrix{
                        1 \ar@{-}[dd]_{a} \ar@{-}[rr]^{d} \ar@{:}[dr]_{b}
                            &*{}
                                &4 \ar@{-}[dd]^{h}
                                    \\
                        *{} 
                            &3 \ar@{-}[ur]_{f} \ar@{:}[dr]^{g}
                                &*{}
                                    \\
                        2  \ar@{-}[ur]^{c} \ar@{:}[rr]_{e}
                            &*{}
                                &5
                    }}
                \]
            \end{block}
        
        \column{.45\textwidth}
            \begin{block}{Простой путь $(1,4)$\\$1,b,3,g,5,h,4$}
                \[
                    {\xymatrix{
                        1 \ar@{<-}[dd]_{a} \ar@{<-}[rr]^{d} \ar@{:>}[dr]_{b}
                            &*{}
                                &4 \ar@{<:}[dd]^{h}
                                    \\
                        *{} 
                            &3 \ar@{->}[ur]_{f} \ar@{:>}[dr]^{g}
                                &*{}
                                    \\
                        2  \ar@{->}[ur]^{c} \ar@{<-}[rr]_{e}
                            &*{}
                                &5
                    }}
                \]
            \end{block}
    \end{columns}
\end{frame}

\begin{frame}
    \frametitle{Цикл и контур}

    \begin{columns}
        \column{.45\textwidth}
            \begin{block}{Цикл \\$1,d,4,h,5,e,2,a,1$}
                \[
                    {\xymatrix{
                        1 \ar@{:}[dd]_{a} \ar@{:}[rr]^{d} \ar@{-}[dr]_{b}
                            &*{}
                                &4 \ar@{:}[dd]^{h}
                                    \\
                        *{} 
                            &3 \ar@{-}[ur]_{f} \ar@{-}[dr]^{g}
                                &*{}
                                    \\
                        2  \ar@{-}[ur]^{c} \ar@{:}[rr]_{e}
                            &*{}
                                &5
                    }}
                \]
            \end{block}
        
        \column{.45\textwidth}
            \begin{block}{Контур\\$1,b,3,g,5,e,2,c,3,f,4,d,1$}
                \[
                    {\xymatrix{
                        1 \ar@{<-}[dd]_{a} \ar@{<:}[rr]^{d} \ar@{:>}[dr]_{b}
                            &*{}
                                &4 \ar@{<-}[dd]^{h}
                                    \\
                        *{} 
                            &3 \ar@{:>}[ur]_{f} \ar@{:>}[dr]^{g}
                                &*{}
                                    \\
                        2  \ar@{:>}[ur]^{c} \ar@{<:}[rr]_{e}
                            &*{}
                                &5
                    }}
                \]
            \end{block}
    \end{columns}    
    Циклы и контуры также могут быть \alert{простыми}, если все проходимые вершины различны.
\end{frame}

\section{Способы задания графов}

\appendix

\section{Библиография}

Библиографию, конечно, можно определить и не используя bibtex!

\begin{frame}
    \frametitle{Источники}
    
    Из книг по \LaTeXe\ можно рекомендовать \cite{bib:cotelnikov,bib:baldin}. Про \TeX\ от автора \cite{bib:knuth:AllAbout}.
\end{frame}


%слайд раскладывается на несколько: allowframebreaks
\begin{frame}[allowframebreaks]{Библиография}

    \bibliographystyle{gost780u}
    \bibliography{bibliobase}
\end{frame}


\end{document}