%Это комментарий
%вначале выполняется общая настройка
%указывается класс документа (влияет на отображение). В CTAN архивах 
%находится огромное множество классов для оформления статей, книг, 
%буклетов, презентаций и сборников...
\documentclass{article} %делаем статью. Хотите книгу? 
                        %Поставьте book вместо article...
\usepackage[utf8]{inputenc} %задаем кодировку исходного текста
\usepackage[russian]{babel} %включаем поддержку русского языка. В частности
                            %корректные переносы слов на новую строку
\usepackage{hyperref}       %включаем гиперссылки в pdf документе

%общие данные для документа
\title{Кодирование печатных документов в \LaTeX} %название
\author{М.~М.~Шихов} %автор
\date{15 декабря 2011 года} %дата. вставка текущей даты: \date{\today}


\begin{document} %начало тела документа

\maketitle %печатает данные титульного листа 
%как именно определяет команда \documentclass см. использование выше

\begin{abstract} %начало аннотации
Это аннотация. Простой пример исходного текста, сравнивая 
который с результатом, легко освоить основы \LaTeX.
\end{abstract} %конец аннотации

%одной строкой формируем содержание
\tableofcontents

\section{Текст заголовка раздела} % помечаем заголовок раздела
%подразделы формируются командами \subsection \subsubsection

Текст раздела. Красивое оформление исходного текста
облегчает работу с ним. Исходные тексты больших документов
можно разнести по нескольким файлам.


\subsection{Текст заголовка подраздела}
\LaTeX имеет замечательные возможности описания текстом математических формул.
%одна или несколько пустых строк обозначают конец абзаца

Вот новый абзац. Вы можете записать формулу прямо в тексте абзаца $a^2=b^2+c^2$.

Вы можете записать формулу отдельной строкой \[a=\sqrt{ b^2+c^2}.\]

Можете пронумеровать её
\begin{equation}
\label{eq:pifagor} %определить понятное имя для метки,
a=\sqrt[2]{ b^2+c^2}  %чтобы затем сослаться на эту формулу
\end{equation}

И затем в любом месте документа сослаться на формулу \ref{eq:pifagor}.

%ссылки на литературу командой \cite{}
Из книг по \LaTeXe\ можно рекомендовать \cite{bib:cotelnikov,bib:baldin}. Про \TeX\ от автора \cite{bib:knuth:AllAbout}. Основной интернет-ресурс \cite{bib:ctan}.


\bibliographystyle{gost780u}
\bibliography{bibliobase}

\end{document} %конец документа
